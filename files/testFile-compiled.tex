% Start of header.
\documentclass{article}
\usepackage[portuguese]{babel}
\usepackage{url}
\usepackage[utf8]{inputenc}
\usepackage{amssymb}
\usepackage[fleqn]{amsmath}
\usepackage{amsthm}

\newcommand{\R}[0]{\mathbb{R}}
\newcommand{\dr}[0]{d \vec{r}}
\newcommand{\dS}[0]{d \vec{S}}
\newcommand{\vF}[0]{\vec{F}}
\newcommand{\sqrBrkts}[1]{\left[#1\right]}
\newcommand{\partialFrac}[2]{\frac{\partial #1}{\partial #2}}
\newcommand{\grad}[1]{\vec{\nabla}#1}
\newcommand{\divergent}[1]{\vec{\nabla} \cdot #1}
\newcommand{\rot}[1]{\textrm{rot} #1}
\newcommand{\ppartialFrac}[3]{\frac{\partial #1}{\partial #2 \partial #3}}

\title{Resolução da 1ª Frequência de AMIII, 2013}
\author{Miguel Murça}
\date{21 Novembro, 2016}
% End of header
% Start of body.

\begin{document}

\maketitle
\section{Exercício 1}

\begin{center}
$ \vF = (e^x, xy)$
\end{center}

\subsection*{a)}

\subsubsection*{Parametrização:}

\begin{gather*}
\vec{r}(t): \\
\qquad x = \cos t\\
\qquad y = \sin t\\
t \in [0, \pi/2]
\end{gather*}

\begin{gather*}
\vec{r'}(t):\\
\qquad x = -\sin t\\
\qquad y = \cos t\\
t \in [0, \pi/2]
\end{gather*}

\subsubsection*{Integral}

\begin{gather*}
\int_C \vF{} \cdot \dr = \int_0^{2 \pi} \vF \cdot \vec{r'}(t) dt =\\
= \int_0^{2\pi} e^{\cos t} \cdot (- \sin t) + \cos t \sin t \cdot \cos t dt = \int_0^{2 \pi} - \sin t e^{\cos t} + \cos^2 t \sin t dt\\
= \sqrBrkts{e^{\cos t}}_0^{2\pi} - 1/3 \sqrBrkts{\cos^3 t}_0^{2\pi} = e^1 - e^1 - 1/3 (1 - 1) = 0
\end{gather*}

\subsection*{b)}

Sejam:
\begin{itemize}
\item $T$ a curva reta de $(1,0)$ para $(0,1)$
\item $T_3$ a curva de $(0,1)$ para $(0,0)$
\item $T_2$ a curva de $(0,0)$ para $(1,0)$
\end{itemize}

O teorema de Green diz-nos que:

\begin{gather*}
\int_t \vF \cdot \dr = \iint_A (\partialFrac{F_y}{x} - \partialFrac{F_x}{x}) dA - \int_{T_2} \vF \cdot \dr - \int_{T_3} \vF \cdot \dr
\end{gather*}

Assim:

\begin{gather*}
\partialFrac{F_y}{y} = y\\
\partialFrac{F_x}{y} = 0
\end{gather*}

\begin{gather*}
\int_{T_2} \vF \cdot \dr = \int_0^1 \vF (t, 0) \cdot <1,0> dt = \int_0^1 e^t dt = e^1 = e
\end{gather*}

\begin{gather*}
\int_{T_3} \vF \cdot \dr = \int_1^0 \vF (0, t) \cdot <0,1> dt = \int_1^0 0 dt = 0
\end{gather*}

\begin{gather*}
\iint_A (\partialFrac{F_y}{x} - \partialFrac{F_x}{y}) dA = \int_0^2 \int_0 ^{1-x} y dy dx = \int_0^1 \frac{(1-x)^2}{2} dx = -\sqrBrkts{\frac{(1-x)^3}{6}}_0^1 = 1/6
\end{gather*}

\vspace{5mm}

Logo:

\begin{center}
$\int_T \vF \cdot \dr = 1/6 - e$
\end{center}

\vspace{5mm}

\subsubsection*{c)}

Não, pois se fosse, $(\partialFrac{F_y}{x} - \partialFrac{F_x}{y}) = 0$.

\vspace{5mm}

\rule{\textwidth}{0.4pt}

\vspace{5mm}

\section{Exercício 2}

\begin{center}
$S = {(x,y,z) \in \R^3 : z^2 = x^2 + y^2, \qquad -1 \le z \le 0}$
\end{center}

Normal exterior.

\begin{center}
$\vF(x,y,z) = <20e^{z^13}, x^{2013z}, z-2013>$
\end{center}

Pelo Teorema da Divergência, e sendo $S_1 : {(x,y,z) \in \R^3: z^2\leq x^2+y^2 \land z=-1}$,
$\tau$ o volume delimitado por $S$, $S_1$

\begin{center}
$\iint_S \vF \cdot \dS = \iiint_\tau \divergent{\vF} d\tau - \iint_{S_1} \vF \cdot \dS$
\end{center}

Assim, em $z=-1$

\vspace{5mm}

\begin{center}
$\vF = <20 e^{-1}, x^{-2013}, -2014>$
\end{center}

\begin{gather*}
\iint_{S_1} \vF \cdot \dS =\\
\int_0^1 \int_0^{2 \pi} \vF(r \cos\theta, r \sin\theta, -1) \cdot <0,0,-1> r d\theta dr =\\
\int_0^1 \int_0^{2 \pi} 2014 r d\theta dr = 2\pi 1007 = 2014 \pi
\end{gather*}

\vspace{5mm}

\begin{gather*}
\divergent{\vF} = \partialFrac{F_x}{x} + \partialFrac{F_y}{y} + \partialFrac{F_z}{z} =\\
= 0+0+1 = 1
\end{gather*}

\vspace{5mm}

\begin{gather*}
\iiint_\tau \divergent{\vF} d\tau = \iiint_\tau 1 d\tau =\\
= V_{\textrm{cone}} = \frac{\pi}{3}
\end{gather*}

\vspace{5mm}

Assim,

\begin{center}
$\iint_S = \vF \cdot \dS = \pi/3 - 2014 \pi = (1/3 - 2014) \pi$
\end{center}

\vspace{5mm}

\rule{\textwidth}{0.4pt}

\vspace{5mm}

\section{Exercício 3}

Se $\vF$ é conservativo, então existe uma função escalar $f$ de
domínio $\R^3$ e conjunto de chegada $\R$ tal que

\begin{center}
$\vF = \grad{f} = <\partialFrac{f}{x}, \partialFrac{f}{y}, \partialFrac{f}{z}>$
\end{center}

Assim,

\begin{gather*}
\rot{\vF} = \vec{\nabla} \times \vF =\\
= <\partialFrac{F_z}{y} - \partialFrac{F_y}{z}, \partialFrac{F_x}{z} - \partialFrac{F_z}{x}, \partialFrac{F_y}{x} - \partialFrac{F_x}{y}>
\end{gather*}

Mas, pelo anteriormente visto,

\begin{gather*}
= <\partialFrac{}{y}\partialFrac{f}{z} - \partialFrac{}{z}\partialFrac{f}{y}, \partialFrac{}{z}\partialFrac{f}{x} - \partialFrac{}{x}\partialFrac{f}{z}, \partialFrac{}{x}\partialFrac{f}{y} - \partialFrac{}{y}\partialFrac{f}{x}>
\end{gather*}

Mas é dado que $\vF$ +e de classe $C^1$, pelo que, uma vez que
$\vF(\partialFrac{f}{x}, \partialFrac{f}{y}, \partialFrac{f}{z})$
e está definido e é contínuo em $\R^3$, então $f$ é de classe $C^2$
em $\R^3$, pelo que

\begin{gather*}
\ppartialFrac{f}{y}{z} = \ppartialFrac{f}{z}{y}\\
\ppartialFrac{f}{z}{x} = \ppartialFrac{f}{x}{z}\\
\ppartialFrac{f}{x}{y} = \ppartialFrac{f}{y}{x}
\end{gather*}

Logo,

\begin{center}
$\rot{\grad{f}} = \rot{\vF} = <0,0,0> = \vec{0}$
\end{center}
\end{document}
% End of body.