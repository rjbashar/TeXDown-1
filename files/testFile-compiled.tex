% Start of header.
\documentclass{article}
\usepackage[utf8]{inputenc}
\usepackage{url}
\usepackage[T1]{fontenc}
\usepackage{amssymb}
\usepackage[fleqn]{amsmath}
\usepackage{amsthm}

\newtheorem{theorem0}{Theorem}
\newtheorem*{theorem1}{Name}
\newtheorem{theorem2}{Lemma}
\newtheorem*{theorem3}{Pythagorean Theorem}
% End of header
% Start of body.

\begin{document}

\begin{theorem0}
    Unnamed theorem.
\end{theorem0}

\vspace{5mm}

\begin{theorem1}
    A named theorem
\end{theorem1}

\vspace{5mm}

\begin{theorem2}
    A lemma
\end{theorem2}

\vspace{5mm}

For example:

\begin{theorem3}
    A rectangle triangle of sides $a$, $b$ and $c$ must be
\end{theorem3}

    such that $a^2 + b^2 = c^2$
\end{document}
% End of body.