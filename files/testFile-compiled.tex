% Start of header.
\documentclass{article}
\usepackage[utf8]{inputenc}
\usepackage{url}
\usepackage[T1]{fontenc}
\usepackage{amssymb}
\usepackage[fleqn]{amsmath}
\usepackage{amsthm}

\newcommand{\braces}[1]{\left ( #1 \right )}
\newcommand{\dpartial}[2]{\frac{\partial #1}{\partial #2}}
\newcommand{\ddpartial}[2]{\frac{\partial^2 #1}{\partial #2^2}}
\newcommand{\position}[0]{\vec{r}}
\newcommand{\velocity}[0]{\vec{v}}
\newcommand{\versor}[1]{\hat{e_{#1}}}

\newtheorem*{theorem0}{Equação de onda what}

\title{Dedução da equação de onda}
\author{}
\date{}
% End of header
% Start of body.

\begin{document}

\maketitle
É condição necessária mas não suficiente para que $\psi$ seja uma função
de onda que

$$\psi = \psi \braces{\position - \velocity t} = \psi \braces{u}$$

$$u := \position - \velocity t = \sum_i q_i \versor{i} - \sum_i \dot{q_i} \versor{i} t$$

\begin{gather*}
\dpartial{\psi}{q_i} = \dpartial{\psi}{u} \dpartial{u}{q_i} = \dpartial{\psi}{u} \cdot 1\\
\Rightarrow \dpartial{\psi}{q_i} = \dpartial{\psi}{u}
\end{gather*}

Por outro lado,

$$ \dpartial{\psi}{t} = \dpartial{\psi}{u} \dpartial{u}{t} = \dpartial{\psi}{u} (- \velocity) $$

Observando a 2ª derivada:

\begin{gather*}
\ddpartial{\psi}{q_i} = \dpartial{}{u} \dpartial{\psi}{q_i} \dpartial{u}{q_i} = \ddpartial{\psi}{u}\\
\ddpartial{\psi}{t} = \dpartial{}{u} \braces{\dpartial{\psi}{t}} \dpartial{u}{t} =\\
= \dpartial{}{u} \braces{\dpartial{\psi}{u} \braces{- \velocity}} \braces{- \velocity} =\\
= v^2 \ddpartial{\psi}{u}
\end{gather*}

Assim, obtém-se

\begin{theorem0}
\begin{center}
$$\ddpartial{\psi}{q_i} = \frac{1}{v^2} \ddpartial{\psi}{t}$$
\end{center}
\end{theorem0}

\vspace{5mm}

Mas isto $\neq \nabla^2 \psi = \frac{1}{v^2} \ddpartial{\psi}{t}$ ?
\end{document}
% End of body.